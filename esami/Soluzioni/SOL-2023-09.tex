\documentclass[a4paper]{article}

\usepackage[T1]{fontenc}
\usepackage[italian]{babel}
\usepackage{amsmath}
\usepackage{amssymb}
\usepackage{listings}
\usepackage[margin=3cm]{geometry}
\usepackage{tikz}

\usetikzlibrary{automata, positioning, arrows}
\tikzset{
  ->, % makes the edges directed
  >=stealth', % makes the arrow heads bold
  node distance=3cm, % specifies the minimum distance between two nodes. Change if necessary.
  every state/.style={thick, fill=gray!10}, % sets the properties for each ’state’ node
  initial text=$ $, % sets the text that appears on the start arrow
}

\title{Soluzioni esame LFC 2023-09}
\date{}

\begin{document}

\maketitle

\section*{Esercizio 1}

\textbf{RISPOSTA:} FALSO

\section*{Esercizio 2}

\begin{figure}[ht]
  \centering
    \begin{tikzpicture}
      \node[state, initial] (s1) {$1$};
      \node[state, accepting, right of=s1] (s2) {$2$};

      \draw (s1) edge[loop above] node{$b$} (s1)
      (s1) edge[above] node{$a,b$} (s2);
    \end{tikzpicture}
\end{figure}
\textbf{RISPOSTA:} 2 stati, 1 finale.
\section*{Esercizio 3}

\begin{figure}[ht]
  \centering
    \begin{tikzpicture}[caption=test]
      \node[state, initial, accepting] (A) {A};
      \node[state, right of=A] (B) {B};
      \node[state, accepting, below of=B] (C) {C};
      \node[state, accepting, below right of=C] (D) {D};
      \node[state, below of=A] (E) {E};
      \node[state, below right of=A] (F) {F};

      \draw (A) edge[above] node{$a$} (B)
      (A) edge[left] node{$a$} (E)
      (B) edge[left] node{$a$} (C)
      (B) edge[above, bend right] node{$a$} (F)
      (D) edge[above, bend right] node{$a$} (B)
      (D) edge[above, bend left] node{$a$} (E)
      (E) edge[above] node{$a$} (C)
      (E)  edge[above, bend left] node{$a$} (F);
    \end{tikzpicture}
  \caption{NFA di partenza}
\end{figure}

Ora eseguo la subset construction per trovare il numero di stati.\\
Le lettere in grassetto sono stati finali.

\begin{figure}[ht]
  \centering
  \begin{tabular}{|c|c|}
    \hline
    & $a$\\
    \hline
    $T_{0}$ = \textbf{A} & $T_{1}$\\
    \hline
    $T_{1}$ = B, E & $T_{2}$\\
    \hline
    $T_{2}$ = \textbf{C}, F & $\emptyset$ \\
    \hline
  \end{tabular}
\end{figure}

\textbf{RISPOSTA:} 3 stati, 2 finali.

\section*{Esercizio 4}

first($S$) = \{$a,b,\varepsilon$\} follow($S$) = \{$\$$\}\\
first($B$) = \{$a,b,\varepsilon$\} follow($B$) = \{$\$$\}

\textbf{RISPOSTA:}
\begin{figure}[ht]
  \centering
  \begin{tabular}{|c|c|c|c|}
    \hline
    & $a$ & $b$ & $\$$\\
    \hline
    $S$
    & \begin{tabular}{@{}c@{}}$S \to aS$ \\ $S \to B$\end{tabular}
    & $S \to B$
    & \begin{tabular}{@{}c@{}}$S \to B$ \\ $S \to \varepsilon$\end{tabular}\\
    \hline
  \end{tabular}
\end{figure}

\section*{Esercizio 5}

\'E necessario prendere il DFA della tabella e aggiungere un pozzo per rendere completa la funzione di transizione.\\
\textbf{RISPOSTA:} MINIMO.

\section*{Esercizio 6}

\begin{minipage}{0.25\textwidth}
  0:
  \begin{tabular}{|c|}
    \hline
    $S^{\prime} \to \cdot S$, \$\\
    \hline
    $S \to \cdot aS$, \$\\
    $S \to \cdot B$, \$\\
    $S \to \cdot$, \$\\
    $B \to \cdot bB$, \$\\
    $B \to \cdot S$, \$\\
    \hline
  \end{tabular}
\end{minipage}
\begin{minipage}{0.25\textwidth}
  $\tau (0,a)$=1
  \begin{tabular}{|c|}
    \hline
    $S \to a \cdot S$, \$\\
    \hline
    $S \to \cdot aS$, \$\\
    $S \to \cdot B$, \$\\
    $S \to \cdot$, \$\\
    $B \to \cdot bB$, \$\\
    $B \to \cdot S$, \$\\
    \hline
  \end{tabular}
\end{minipage}
\begin{minipage}[c]{0.25\textwidth}
  $\tau (1,a) = 1$
\end{minipage}
\begin{minipage}{0.25\textwidth}
  $\tau (1,S)$=2
  \begin{tabular}{|c|}
    \hline
    $S \to aS \cdot $, \$\\
    $B \to S  \cdot$, \$\\
    \hline
  \end{tabular}
\end{minipage}

\textbf{RISPOSTA:} conflitto r/r in $T[I[\![aaS]\!], \$]$ tra $S \to aS$ e $B \to S$.

\section*{Esercizio 7}

Se già la tabella LR(1) ha un conflitto possiamo dire che non è LALR(1).\\
\textbf{RISPOSTA:} NON LALR.

\section*{Esercizio 8}

\textbf{RISPOSTA:} $[P[\![BaB]\!],a]$ è da risolvere in favore della reduce.

\section*{Esercizio 9}

\textbf{RISPOSTA:}
\begin{lstlisting}
  L.array = table.get(id)
  L.width = L.array_ewidth
  L.addr = newtemp()
  gen(L.addr '=' E.addr '*' L.width)
\end{lstlisting}

\section*{Esercizio 10}

\begin{figure}
  \centering
    \begin{tikzpicture}
      \node[state] (A) {2};
      \node[state, below right of=A] (B) {1};
      \node[state, below left of=B] (C) {3};
      \node[state, below left of=C] (D) {5};
      \node[state, below right of=B] (E) {4};
      \node[state, below right of=E] (F) {5};

      \draw (A) edge[] node{} (B)
      (B) edge[] node{} (C)
      (C) edge[] node{} (D)
      (B) edge[] node{} (E)
      (E) edge[] node{} (F);
    \end{tikzpicture}
\end{figure}

Disegno dell'albero da valutare nella prossima pagina.\\
\textbf{RISPOSTA:} 541352

\end{document}
