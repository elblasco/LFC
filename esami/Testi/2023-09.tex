\documentclass[a4paper]{article}

\usepackage[T1]{fontenc}
\usepackage[italian]{babel}
\usepackage{amsmath}
\usepackage{amssymb}
\usepackage{listings}
\usepackage[margin=3cm]{geometry}

\title{Esame LFC 2023-09}
\date{}

\begin{document}

\maketitle

\section*{Esercizio 1}

``Se una grammatica è LR(1) allora è anche LALR(1).'' se la frase è vera scrivere \textbf{VERO} altrimenti \textbf{FALSO}.

\section*{Esercizio 2}

Sia $r = a | (\varepsilon | b)(\varepsilon | b)^{*}a$ un espressione regolare e $\mathcal{D}$ il DFA minimo tale che $L(r) = L(\mathcal{D})$, quanti stati ha $\mathcal{D}$? Quanti di questi sono finali?

\section*{Esercizio 3}

Sia $\mathcal{N}$ un NFA con stato iniziale A e stati finali \{A, C, D\}.
\begin{center}
  $\mathcal{N}$:
  \begin{tabular}{|c|c|}
    \hline
    & a\\
    \hline
    A & \{B, E\}\\
    \hline
    B & \{C, F\}\\
    \hline
    C & $\emptyset$\\
    \hline
    D & \{B, E\}\\
    \hline
    E & \{C, F\}\\
    \hline
    F & $\emptyset$\\
    \hline
  \end{tabular}
\end{center}
Dire quanti stati ha il DFA ottenuto con subset construction da $\mathcal{N}$, quanti di questi stati sono finali?

\section*{Esercizio 4}

\begin{equation*}
  \mathcal{G}_{235}:
  \begin{cases}
    &S \to aS | B | \varepsilon\\
    &B \to bB | S
  \end{cases}
\end{equation*}
Scrivere la riga della tabella di parsing LL(1) per il non terminale S.

\section*{Esercizio 5}

Sia $\mathcal{D}$ un DFA con stato iniziale A e stato finale B.
\begin{center}
  $\mathcal{D}$:
  \begin{tabular}{|c|c|c|}
    \hline
    & a & b\\
    \hline
    A & B & C\\
    \hline
    B & D & C\\
    \hline
    C & D & $\emptyset$\\
    \hline
    D & C & B\\
    \hline
  \end{tabular}
\end{center}
Scrivere ``\textbf{MINIMO}'' se $\mathcal{D}$ è già minimo, altrimenti scrivere quanti stati ha il DFA minimo e quanti di questi sono finali.

\section*{Esercizio 6}

Sia $\mathcal{A}$ l'automa caratteristico per il paringLR(1) di $\mathcal{G}_{235}$, $I$ lo stato iniziale di $\mathcal{A}$, $T$ la tabella di parsing LR(1) per $\mathcal{G}_{235}$. Se $T$ non contiene alcun conflitto nello stato $I[\![aaS]\!]$, rispondere ``NO CONFLICT''. Altrimenti, per ciascuna $X$ tale che $T[I[\![aaS]\!], X]$ contiene un conflitto, dire, specificando quale $X$ si fa riferimento: (i) di che tipo di conflitto si tratta; (ii) quale/i riduzione/i sono coinvolte.

\section*{Esercizio 7}

Se $\mathcal{G}_{235}$ è una grammatica LALR(1) scrivere ``\textbf{LALR}'' altrimenti scrivere ``\textbf{NON LALR}''.

\section*{Esercizio 8}

\begin{equation*}
  S_{235}:
  \begin{cases}
    S \to B & \{S.v = B.v\}\\
    B \to t & \{B.v = true\}\\
    B \to f & \{B.v = false\}\\
    B \to B_{1}\ i\ B_{2} & \{B.v = (not\ B_{1}.v)\ or\ B_{2}.v\}\\
    B \to B_{1}\ a\ B_{2} & \{B.v = B_{1}.v\ and\ B_{2}.v\}\\
    B \to n\ B_{1} & \{B.v = not\ B_{1}.v\}
  \end{cases}
\end{equation*}
Il praser LALR(1) per $S_{235}$ ha 6 conflitti in $[P[\![n\ B]\!], i]$, $[P[\![n\ B]\!], a]$, $[P[\![B\ i\ B]\!], i]$, $[P[\![B\ i\ B]\!], i]$, $[P[\![B \ a\ B]\!], i]$ e in $[P[\![B\ a\ B]\!], i]$. Per rendere l'operatore $a$ associativo a sinistra che conflitto/i devo risolvere? Con che operazione?

\section*{Esercizio 9}

\begin{equation*}
  \begin{cases}
    S \to id\ =\ E & \{gen(table.get(id)\ '='\ E.addr)\}\\
    L \to id\ [\ E\ ]
  \end{cases}
\end{equation*}
Completare la grammatica con le regole da associare alla seconda produzione, la grammatica è l'esempio visto in classe per l'indirzzamento di array in row major order.

\section*{Esercizio 10}

\begin{equation*}
  \begin{cases}
    R \to S & \{eval(S.n)\}\\
    S \to a\ A\ B & \{S.n = newNode(1, A.n, B.n)\}\\
    S \to a\ d\ S_{1} & \{S.n = newNode(2, null, S_{1}.n)\}\\
    A \to C\ a & \{A.n = newNode(3, C.n, null)\}\\
    B \to C\ b & \{B.n = newNode(4, null, C.n)\}\\
    C \to \varepsilon & \{C.n = newNode(5, null, null)\}
  \end{cases}
\end{equation*}
La funzione $newNode(val,sx, dx)$ crea un nodo il valore è $val$ e $sx, dx$ sono rispettivamente i figli destro e sinistro. La funzione $eval(N)$ è definita come:
\begin{lstlisting}[mathescape=true]
  eval(N){
    if (N.dx $\ne$ null) eval(N.dx)
    print(N.val)
    if (N.sx $\ne$ null) eval(N.sx)
  }
\end{lstlisting}
Se la parola $adaab$ non appartiene al linguaggio scrivere ``\textbf{ERRORE}'' altrimenti scrivere cosa ritorna la funzine \texttt{eval}.

\end{document}
